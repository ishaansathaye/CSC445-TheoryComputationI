\documentclass{article}
\def\showanswers{0}
\def\whichlecture{}
\def\lecturetopic{}

%%%% Setup
% \documentclass{article}
\usepackage{amsmath}
\usepackage{amssymb}
\usepackage{amsthm}
\usepackage{caption}
\usepackage{enumerate}
\usepackage[shortlabels]{enumitem}
\usepackage[pdftex]{graphicx}
\usepackage[scaled]{helvet}
\usepackage{hyperref}
\usepackage{listings}
\usepackage{lstautogobble}
\usepackage{tcolorbox}
\usepackage{tikz}
\usetikzlibrary{automata, positioning, arrows}
\usetikzlibrary{shapes.geometric}
\usetikzlibrary{calc,shapes.multipart,chains,arrows}
\tikzstyle{vertex}=[circle,draw=black,minimum size=20pt,inner sep=0pt]
\tikzstyle{edge} = [draw,thick, ->]
\tikzstyle{undirected edge} = [draw,thick,-]
\tikzstyle{weight} = [font=\small]


\renewcommand\familydefault{\sfdefault} 
\usepackage[T1]{fontenc}


\theoremstyle{definition}
\newtheorem{exercise}{Exercise}
\newtheorem{question}{Question}

\newenvironment {solution}
{
\begin{tcolorbox}
}
{
\end{tcolorbox}
}
%%%%


\title{Problem Set 1B: Regular Languages}
\author{Daniel Frishberg}
\date{Cal Poly CSC 445, Fall 2024}

\begin{document}
\lstset{upquote=true}
\setlength\parindent{0em}

\maketitle
\section{Instructions for Submission}
The \textbf{regular deadline} for this problem set is \textbf{Monday, April 29, at 11:59pm.} The \textbf{late deadline} for this problem set is \textbf{Tuesday, April 30, at 11:59pm.} Please upload a PDF or image of your work to Canvas. Legible handwriting or typing is fine. Please submit \textbf{just your answers, and not the problem set document}.

\section{How to solve/use this problem set}
Some questions are graded on effort. These require a credible, reasonable attempt for credit. Any such attempt will receive full credit. Other questions are graded on correctness.
\textbf{Questions that are graded on correctness are indicated as such in bold.}

\section{Required Problems}
\begin{enumerate}
    \item \textbf{(Graded on correctness) (20 points)} Construct an NFA that recognizes the language defined by the regular expression
    $$(ab(a \cup b))^*$$
    \textbf{Use the regex-to-NFA conversion procedure from the ``Equivalences'' lecture in class. You just need to draw the final NFA obtained via this procedure, with all resulting states and transitions.}

    \begin{solution}
        \begin{tikzpicture}[shorten >=1pt, node distance=2cm, on grid, auto]
            \node[state, initial, accepting] (q_0)   {$q_0$}; 
            \node[state] (q_1) [right=of q_0] {$q_1$}; 
            \node[state] (q_2) [right=of q_1] {$q_2$}; 
            \node[state, accepting] (q_3) [right=of q_2] {$q_3$};
         
             \path[->]
             (q_0) edge node {a} (q_1)
             (q_1) edge node {b} (q_2)
             (q_2) edge node {a, b} (q_3)
             (q_3) edge [bend left] node {a} (q_1); % loop to repeat the pattern
         \end{tikzpicture}
    \end{solution}

    \item \textbf{(Graded on correctness) (20 points)} Consider the following NFA.
    
    \begin{tikzpicture}[node distance=7em]
        \node[state, initial] (q0) {$q_0$};
        \node[state, right of=q0] (q1) {$q_1$};
        \node[state, accepting, below of=q0] (q2) {$q_2$};
        \draw (q0) edge[loop above] node{a} (q0)
        (q0) edge[bend left, style = ->] node [above] {a} (q1)
        (q1) edge[bend left, style = ->] node [above] {b} (q0)
        (q1) edge[loop above] node [above] {b} (q1)

        (q1) edge[style = ->] node [right] {b} (q2)
        (q2) edge[style = ->] node [left] {$\varepsilon$} (q0)
        ;
    \end{tikzpicture} 
        
    Let $L$ be the language this NFA recognizes. Construct a DFA that recognizes $L$, \textbf{using the procedure shown in the ``Equivalences'' lecture.} You can just draw the final DFA, but \textbf{do label the states in the DFA as subsets of the states in the NFA.} \textbf{You do not have to draw unreachable or dead states.}

    \begin{solution}
        \begin{tikzpicture}[node distance=7em]
            \node[state, initial] (S0) {$S_0$}; % Start state
            \node[state, right of=S0] (S1) {$S_1$};  
            \node[state, accepting, right of=S1] (S2) {$S_2$};  
        
            \draw (S0) edge[->] node[above]{a} (S1)
        
                  (S1) edge[loop above] node{a} (S1)
                       edge[->] node[below]{b} (S2)
        
                  (S2) edge[loop above] node{b} (S2)
                       edge[->, bend right=30] node[above]{a} (S1); % Curved transition from S2 to S1
        \end{tikzpicture}
    \end{solution}

    Where $S_0 = \{q_0\}$, $S_1 = \{q_0, q_1\}$, and $S_2 = \{q_0, q_1, q_2\}$.
    
    
    \item \textbf{(Graded on effort) (20 points)} Convert the DFA you obtained in your answer to Question 2 to a regular expression using the procedure shown in the lecture ``Equivalences''.
    \textbf{Show each intermediate machine in the procedure.}

    Add start and end states:

    \begin{tikzpicture}[node distance=7em]
        \node[state, initial] (start) {start};
        \node[state, right of=start] (s0) {$S_0$};  
        \node[state, right of=s0] (s1) {$S_1$};  
        \node[state, right of=s1] (s2) {$S_2$};  
        \node[state, accepting, below of=s2] (end) {end}; % Position end below S2
        
        \draw (start) edge[->] node[above] {$\varepsilon$} (s0)
                (s0) edge[->] node[above]{a} (s1)
    
                (s1) edge[loop above] node{a} (s1)
                    edge[->] node[below]{b} (s2)
    
                (s2) edge[loop above] node{b} (s2)
                    edge[->, bend right=30] node[above]{a} (s1)
                    edge[->, bend right=60] node[left] {$\varepsilon$} (end); % Curved transition downwards
    \end{tikzpicture}

    Remove state $S_2$:

    \begin{tikzpicture}[node distance=7em]
        \node[state, initial] (start) {start};
        \node[state, right of=start] (s0) {$S_0$};  
        \node[state, right of=s0] (s1) {$S_1$};  
        \node[state, accepting, right of=s1] (end) {end};
    
        \draw (start) edge[->] node[above] {$\varepsilon$} (s0)
              (s0) edge[->] node[above] {a} (s1)
    
              (s1) edge[loop above] node{a} (s1)
                   edge[->] node[above] {$b(a \cup b^*)$} (end);
    \end{tikzpicture}

    Remove states $S_0$ and $S_1$:

    \begin{solution}
        \begin{tikzpicture}[node distance=10em]
            \node[state, initial] (start) {start};
            \node[state, accepting, right of=start] (end) {end};
        
            \draw (start) edge[->] node[above] {$a(a^*)b(a \cup b^*)$} (end);
        \end{tikzpicture}
    \end{solution}

    Final regular expression: $a(a^*)b(a \cup b^*)$.

    % DFA To regex
    \item  \textbf{(Graded on correctness) (20 points)} 
    Let $L_1 = \{a^kb^l \mid k \geq l \geq 0\}$.
    Prove that the langauge $L$ is not regular using the Pumping Lemma.

    \begin{solution}
        Suppose $L_1$ is regular. Let $p$ be the pumping length given by the Pumping Lemma. Consider the string $w = a^pb^p$. Since $w \in L_1$ and $|w| \geq p$, the Pumping Lemma guarantees that $w$ can be split into three parts, $w = xyz$, such that:
        \begin{enumerate}
            \item $|xy| \leq p$
            \item $|y| > 0$
            \item For all $i \geq 0$, $xy^iz \in L_1$
        \end{enumerate}
        Since $|xy| \leq p$, $y$ consists of only $a$'s. Let $w' = xy^0z = xz$. Then $w'$ has more $b$'s than $a$'s, so $w' \notin L_1$. This contradicts the Pumping Lemma, so $L_1$ is not regular.
    \end{solution}

    \item \textbf{(Graded on correctness) (20 points)}
    Let $L = \{w \mid \textnormal{ the number of a's in w is } \geq \textnormal{ the number of b's }\}$. Use the fact that $L_1$ (from part (4)) is not regular, along with closure properties, to prove that $L$ is not regular.

    \begin{solution}
        Suppose $L$ is regular. 
        
        Let $L_1 = \{a^kb^l \mid k \geq l \geq 0\}$, from part (4).
        
        Let $L_2 = a^*b^*$ and is regular. 
        
        $L \cap L_2 = L_1$ is regular by closure properties.
        
        By closure properties, $L_1$ is regular.

        However, we know that $L_1$ is not regular (contradiction). 
        
        Therefore, $L$ is not regular.
    \end{solution}



\end{enumerate}

\end{document}

