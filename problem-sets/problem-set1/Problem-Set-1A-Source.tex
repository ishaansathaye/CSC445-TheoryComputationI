\documentclass{article}
\def\showanswers{0}
\def\whichlecture{}
\def\lecturetopic{}

%%%% Setup
% \documentclass{article}
\usepackage{amsmath}
\usepackage{amssymb}
\usepackage{amsthm}
\usepackage{caption}
\usepackage{enumerate}
\usepackage[shortlabels]{enumitem}
\usepackage[pdftex]{graphicx}
\usepackage[scaled]{helvet}
\usepackage{hyperref}
\usepackage{listings}
\usepackage{lstautogobble}
\usepackage{tcolorbox}
\usepackage{tikz}
\usetikzlibrary{automata, positioning, arrows}
\usetikzlibrary{shapes.geometric}
\usetikzlibrary{calc,shapes.multipart,chains,arrows}
\tikzstyle{vertex}=[circle,draw=black,minimum size=20pt,inner sep=0pt]
\tikzstyle{edge} = [draw,thick, ->]
\tikzstyle{undirected edge} = [draw,thick,-]
\tikzstyle{weight} = [font=\small]


\renewcommand\familydefault{\sfdefault} 
\usepackage[T1]{fontenc}


\theoremstyle{definition}
\newtheorem{exercise}{Exercise}
\newtheorem{question}{Question}

\newenvironment {solution}
{
\begin{tcolorbox}
}
{
\end{tcolorbox}
}
%%%%


\title{Problem Set 1A}
\author{Daniel Frishberg}
\date{Cal Poly CSC 445, Fall 2024}

\begin{document}
\lstset{upquote=true}
\setlength\parindent{0em}

\maketitle
\section{Instructions for Submission}
The \textbf{regular deadline} for this problem set is \textbf{Thursday, October 3, at 11:59pm.} The \textbf{late deadline} for this problem set is \textbf{Friday, October 4, at 11:59pm.} Please upload a PDF or image of your work to Canvas. Legible handwriting or typing is fine. Please submit \textbf{just your answers, and not the problem set document}.

\section{How to solve/use this problem set}
The required questions are designed to help you develop and sharpen the problem-solving tools from class. The practice questions are intended to help you gain more practice with the building blocks from lecture. 
\\\\
Some questions are graded on effort. These require a credible, reasonable attempt for credit. Any such attempt will receive full credit. Other questions are graded on correctness.
\textbf{Questions that are graded on correctness are indicated as such in bold.}

\section{Required Problems}
\begin{enumerate}
    \item \textbf{(Graded on correctness) (25 points)} Given a bit string (a string of 1's and 0's) $w$, the \emph{parity} of $w$ is defined to be 1 if $w$ has an odd number of 1's, and 0 if $w$ has an even number of 1's. Give a recursive definition for the parity of a bit string $w$, with $\varepsilon$ (the empty string) as the base case. (See the lecture notes on string length for inspiration.)
    \begin{itemize}
        \item $\varepsilon$ has parity 0, since it has an even number of 1's (0 of them): $parity(\varepsilon) = 0$.
        \item If $w \in L$ and has parity $p$, then $w0$ has parity $p$, since number of 1's has not changed, so $parity(w0) = parity(w) = p$.
        \item If $w \in L$ and has parity $p$, then $w1$ has parity $1-p$, since the number of 1's has changed by 1, so $parity(w1) = 1 - parity(w) = 1 - p$.
    \end{itemize}
    \item Give a recursive definition for each of the following languages:
    \begin{enumerate}[(a)]
        \item \textbf{(Graded on effort) (10 points)} The set of all strings of parentheses over the alphabet $\Sigma = \{(, )\}$ that are balanced.
            \begin{solution}
                \begin{itemize}
                    \item The empty string $\varepsilon$ is balanced and so $\varepsilon \in L$.
                    \item If $x \in L$ and $y \in L$, where $x$ and $y$ are balanced strings, then $(x)y \in L$.
                \end{itemize}
		\end{solution}
        \item \textbf{(Graded on effort) (15 points)} The set of all bit strings over the alphabet $\Sigma = \{0, 1\}$ \textbf{not} containing the string $11$.
 		\begin{solution}
	      \begin{itemize}
            \item $\varepsilon \in L$, 0 $\in L$, 1 $\in L$
            \item If $w \in L$, then $0w \in L$ and $10w \in L$
          \end{itemize}
		\end{solution}
    \end{enumerate}

    \item Give a mathematical set-builder definition for each of the following languages. Set-builder notation means, for example, an answer such as
    $$L = \{a^{3k} \mid k \geq 0\}$$
    or $$L = \{(aaa)^k \mid k \geq 0\}.$$
    \textbf{Do not use a description in words.} 
    \begin{enumerate}[(a)]
        \item \textbf{(Graded on effort) (10 points)} The set of all strings over the alphabet $\Sigma = \{a\}$ whose length is equal to a multiple of 3 plus a multiple of 5.
	      \begin{solution}
	      $$L = \{a^{3k + 5l} \mid k, l \geq 0\}$$
		\end{solution}
        \item \textbf{(Graded on correctness) (15 points)} The set of all strings over the alphabet $\Sigma = \{a, b\}$ with exactly twice as many $b$'s as $a$'s, and with no occurrences of the substring $ba$.        
	      \begin{solution}
	      $$L = \{a^k b^{2k} \mid k \geq 0\}$$
		\end{solution}
    \end{enumerate}

    \item Prove the following languages are regular by providing a DFA that recognizes each:
    \begin{enumerate}[(a)]
        \item \textbf{(Graded on effort) (10 points)} $$L_1 = \{w \in \Sigma^* \mid w \textnormal{ does not contain the substring }00 \}$$
    where $\Sigma = \{0, 1\}.$ 
	      \begin{solution}
            \begin{tikzpicture}[shorten >=1pt, node distance=2cm, on grid, auto]

                \node[state, accepting, initial] (q0)   {$q_0$};
                \node[state, accepting, right of=q0] (q1)   {$q_1$};
                \node[state, accepting, right of=q1] (q2)   {$q_2$};
             
                \path[->]
                (q0) edge [bend left] node {0} (q1)
                     edge [loop above] node {1} ()
             
                (q1) edge [bend left] node {1} (q2)
                     
                (q2) edge [bend left] node {0} (q1)
                     edge [loop above] node {1} ();
             
             \end{tikzpicture}
		\end{solution}
        \item \textbf{(Graded on correctness)(15 points)} $$L_2 = \{w \in \Sigma^* \mid w \textnormal { contains at least two 0's and does not contain the substring }00\}$$
        where $\Sigma = \{0, 1\}.$ 
    
        Note that $L_2 \subseteq L_1$ i.e. $L_2$ is a subset of $L_1$. 
	      \begin{solution}
            \begin{tikzpicture}[shorten >=1pt, node distance=2cm, on grid, auto]

                \node[state, initial] (q0)   {$q_0$};
                \node[state, right of=q0] (q1)   {$q_1$};
                \node[state, right of=q1] (q2)   {$q_2$};
                \node[state, accepting, right of=q2] (q3)   {$q_3$};

                \path[->]
                (q0) edge [loop above] node {1} ()
                        edge [bend left] node {0} (q1)

                (q1) edge [bend left] node {1} (q2)
                
                (q2) edge [loop above] node {1} ()
                        edge [bend left] node {0} (q3)
                        
                (q3) edge [bend left] node {1} (q0);

            \end{tikzpicture}
		\end{solution}
    \end{enumerate}


\end{enumerate}

\end{document}

