\documentclass{article}
\def\showanswers{0}
\def\whichlecture{}
\def\lecturetopic{}

%%%% Setup
\usepackage{amsmath}
\usepackage{amssymb}
\usepackage{amsthm}
\usepackage{caption}
\usepackage{enumerate}
\usepackage[shortlabels]{enumitem}
\usepackage[pdftex]{graphicx}
\usepackage[scaled]{helvet}
\usepackage{hyperref}
\usepackage{listings}
\usepackage{lstautogobble}
\usepackage{tcolorbox}
\usepackage{tikz}
\usetikzlibrary{automata, positioning, arrows}
\usetikzlibrary{shapes.geometric}
\usetikzlibrary{calc,shapes.multipart,chains,arrows}
\tikzstyle{vertex}=[circle,draw=black,minimum size=20pt,inner sep=0pt]
\tikzstyle{edge} = [draw,thick, ->]
\tikzstyle{undirected edge} = [draw,thick,-]
\tikzstyle{weight} = [font=\small]


\renewcommand\familydefault{\sfdefault} 
\usepackage[T1]{fontenc}


\theoremstyle{definition}
\newtheorem{exercise}{Exercise}
\newtheorem{question}{Question}

\newenvironment {solution}
{
\begin{tcolorbox}
}
{
\end{tcolorbox}
}
%%%%
\newcommand{\qacc}{q_{\textnormal{accept}}}
\newcommand{\qrej}{q_{\textnormal{reject}}}


\title{Problem Set 3: Turing Machines}
\author{Daniel Frishberg}
\date{Cal Poly CSC 445, Fall 2024}

\begin{document}
\lstset{upquote=true}
\setlength\parindent{0em}

\maketitle
\section{Instructions for Submission}
The \textbf{regular deadline} for this problem set is \textbf{Friday, November 22, at 11:59pm.} The \textbf{late deadline} for this problem set is \textbf{Sunday, December 1, at 11:59pm.} Please upload a PDF or image of your work to Canvas. Legible handwriting or typing is fine. Please submit \textbf{just your answers, and not the problem set document}.

\section{How to solve/use this problem set}
Some questions are graded on effort. These require a credible, reasonable attempt for credit. Any such attempt will receive full credit. Other questions are graded on correctness.
\textbf{Questions that are graded on correctness are indicated as such in bold.}


\section{Required Problems}
\begin{enumerate}
    \item  (25 points) \textbf{(Graded on correctness)} Give an implementation-level description (not state diagrams) of Turing machine that recognizes the following language:
       
        $\{s\#t \mid s,t \in \{0,1\}^* \textnormal{ and } s \textnormal{ is a substring of } t\}$.
        
        \textit{(Hint: try writing high-level pseudocode for an algorithm, then translate it into implementation-level language.)}

        \begin{solution}
            The following machine $M$ recognizes the language $\{s\#t \mid s,t 
            \in \{0,1\}^* \textnormal{ and } s \textnormal{ is a substring of } t\}$.

            On input string $w$:
            \begin{enumerate}
                \item Scan the tape from left to right to locate the first $\#$.
                If no $\#$ is found, reject. Mark the position of the $\#$.
                \item Identify s as the substring of $w$ to the left of the $\#$
                and t as the substring of $w$ to the right of the $\#$. Place 
                a marker at the beginning of $s$ and $t$.
                \item For each character in $t$, compare $s$ to the substring of
                $t$ starting at the current position and repeat the following:
                \begin{enumerate}
                    \item Move s marker to the beginning of $s$.
                    \item Compare each character of $s$ to the substring of $t$
                    one character at a time. If characters match then move
                    both markers to the right. If there is a mismatch, move the
                    $t$ marker to the right and repeat the comparison.
                    \item If all characters of $s$ match the substring of $t$,
                    accept.
                \end{enumerate}
                \item Move the $t$ marker to the right and repeat comparison. If
                the end of $t$ is reached and no match is found, reject.
            \end{enumerate}
        \end{solution}

    \item (10 points) \textbf{(Graded on effort)} Consider the following TM state diagram from the first lecture on Turing machines (also from Sipser, Example 3.9, Figure 3.10, p. 173:
    
    \begin{tikzpicture}[node distance=6.5em]
        \node[state, initial] (q1) {$q_1$};
        \node[state, right of=q1] (q8) {$q_8$};
        \node[state, above of=q8] (q2) {$q_2$};
        \node[state, below of=q8] (q3) {$q_3$};
        \node[state, right of=q2] (q4) {$q_4$};
        \node[state, accepting, right of=q8] (qacc) {$\qacc$};
        \node[state, right of=q3] (q5) {$q_5$};
        \node[state, right of=qacc] (q6) {$q_6$};
        \node[state, right  of=q6] (q7) {$q_7$};

        \draw (q1) edge[style = ->] node[left, scale=0.8]{$0 \rightarrow x, R$} (q2)
         (q1) edge[style = ->] node[above, scale=0.8]{$\# \rightarrow R$} (q8)
         (q1) edge[style = ->] node[left, scale=0.8]{$1\rightarrow x, R$} (q3)
        
        (q2) edge[loop above] node[above, scale=0.8]{$0,1\rightarrow R$} (q2)
        (q2) edge[style = ->] node[above, scale=0.8]{$\#\rightarrow R$} (q4)
    
        (q8) edge[loop above] node[above, scale=0.8]{$x\rightarrow R$} (q8)
        (q8) edge[style = ->] node[above, scale=0.8]{$\_ \rightarrow R$} (qacc)
    
        (q3) edge[loop above] node[above, scale=0.8]{$0,1\rightarrow R$} (q3)
        (q3) edge[style = ->] node[above, scale=0.8]{$\# \rightarrow R$} (q5)
    
        (q4) edge[loop above] node[above, scale=0.8]{$x\rightarrow R$} (q4)
        (q4) edge[style = ->] node[left, scale=0.8]{$0 \rightarrow x, L$} (q6)
    
        (q5) edge[loop above] node[above, scale=0.8]{$x\rightarrow R$} (q5)
        (q5) edge[style = ->] node[right, scale=0.8]{$1 \rightarrow x,L$} (q6)
    
        (q6) edge[loop above] node[above, scale=0.8]{$0,1,x\rightarrow L$} (q6)
        (q6) edge[style = ->] node[above, scale=0.8]{$\# \rightarrow L$} (q7)
    
        (q7) edge[loop above] node[above, scale=0.8]{$0,1\rightarrow L$} (q7)
        (q7) edge[style = ->, bend left=100] node[below, scale=0.8]{$x\rightarrow R$} (q1)
        ;
    \end{tikzpicture}    

    Draw the first 10 configurations that this machine goes through when reading the input $0010\#0010$. \textit{(Hint: see bottom of p. 172 for examples of configurations from a different Turing machine).}

        \begin{solution}
            \begin{enumerate}
                \item $q_1 0010\#0010$
                \item $xq_2 010\#0010$
                \item $x0q_2 10\#0010$
                \item $x00q_2 \#0010$
                \item $x00\#q_4 0010$
                \item $x00\#xq_6 010$
                \item $x00\#q_6 x010$
                \item $x00q_6 \#x010$
                \item $x0q_6 0x010$
                \item $xq_6 00x010$
            \end{enumerate}
        \end{solution}

    \item (15 points) \textbf{(Graded on correctness)} Give examples of languages that are\dots 
    \begin{enumerate}[(a)]
        \item Context free but not regular
        \item Decidable but not context free
        \item TM-recognizable but not decidable
        \item Not TM-recognizable
    \end{enumerate}

        \begin{solution}
            (a) $L = \{a^nb^n \mid n \geq 0\}$ is context-free but not regular.

            (b) $L = \{a^nb^nc^n \mid n \geq 0\}$ is decidable but not context-free.

            (c) $EQ_{CFG} = \{\langle G, H\rangle \mid \textnormal{ CFGs } G,H \textnormal{ generate the same language }\}$ is TM-recognizable but not decidable.

            (d) $\overline{A_{TM}} = \{\langle M, w\rangle \mid M \textnormal{ is a TM and } M \textnormal{ does not accept } w\}$ is not TM-recognizable.
        \end{solution}


    \item (25 points) \textbf{(Graded on correctness)} (Sipser Exercise 4.3) Show that the following language is decidable:
    % \begin{enumerate}[(a)]
    %     \item  
        
        $ALL_{DFA} = \{\langle A\rangle \mid A \textnormal{ is a DFA and } L(A) = \Sigma^*\}$
        
        (Hint: Use one of the languages that we have seen in class is decidable, along with a closure property of regular languages.)

        \begin{solution}
            We know that $E_{DFA} = \{\langle A\rangle \mid A \textnormal{ is a 
            DFA and } L(A) = \emptyset\}$ is decidable. We also know that 
            decidable languages are closed under complement. 
            Since $ALL_{DFA} = \overline{E_{DFA}}$, because $\forall w \in
            \Sigma^*$, we know that $w \notin E_{DFA}$. So since $E_{DFA}$ is
            decidable and decidable languages are closed under complement,
            $ALL_{DFA}$ is decidable.
        \end{solution}


    \item (25 points) \textbf{(Graded on correctness)} (Sipser Exercises 5.1-5.2) Consider the languages 
        \[ALL_{CFG} = \{\langle G\rangle \mid G \textnormal{ is a CFG and } L(G) = \Sigma^*\}\]
        and
        \[        EQ_{CFG} = \{\langle G, H\rangle \mid \textnormal{ CFGs } G,H \\
         \textnormal{ generate the same language }\}
        \]

    $ALL_{CFG}$ is undecidable (the proof is in the Sipser textbook).
    
    Prove that $EQ_{CFG}$ is\dots
    
    \begin{enumerate}[(a)]
        \item Undecidable (use the fact that $ALL_{CFG}$ is undecidable)
        \item Co-TM-recognizable
        \item Not TM-recognizable
    \end{enumerate}

    \begin{solution}
        (a) Suppose $EQ_{CFG}$ is decidable. Then there exists a TM $Q$ that 
        decides $EQ_{CFG}$. Construct the following TM, $D$, that decides 
        $ALL_{CFG} = \{\langle G\rangle \mid G \textnormal{ is a CFG and } L(G) = \Sigma^*\}$.
        High-level description of $D$:
            
        On input $\langle G \rangle$:
        \begin{enumerate}
            \item Construct a CFG $H$ such that $L(H) = \Sigma^*$.
            \item Run $Q$ on input $\langle G, H \rangle$.
            \item If $Q$ accepts, accept; else reject.
        \end{enumerate}
        $D$ accepts $\langle G \rangle$ if $L(G) = L(H) = \Sigma^*$, else $D$
        rejects $\langle G \rangle$. $\therefore D$ decides $ALL_{CFG}$.
        (contradiction)
    \end{solution}

    \begin{solution}
        (b) Construct the following TM $M$ that recognizes 
        $\overline{EQ_{CFG}} = \{\langle G, H\rangle \mid L(G) \neq L(H)\}$.
        
        High-level description of $M$:
        
        On input $\langle G, H \rangle$:
        \begin{enumerate}
            \item Construct a CFG $I$ such that $L(I) = \Sigma^*$.
            \item Run $Q$ on input $\langle G, I \rangle$.
            \item If there exists a string $w \in \Sigma^*$ such that $w \in L(G)$
            and $w \notin L(H)$ or $w \notin L(G)$ and $w \in L(H)$, accept; else reject.
        \end{enumerate}
        $\therefore M$ recognizes $\overline{EQ_{CFG}}$.
    \end{solution}

    \begin{solution}
        (c) Suppose for a contradiction that $EQ_{CFG}$ is TM-recognizable.
        Then there exists a TM $M$ that recognizes $EQ_{CFG}$. If $EQ_{CFG}$ is
        TM-recognizable, and $\overline{EQ_{CFG}}$ is co-TM-recognizable, then
        both $EQ_{CFG}$ and $\overline{EQ_{CFG}}$ would be TM-recognizable.
        If both a language and its complement are TM-recognizable, then the
        language is decidable. However, we know that $EQ_{CFG}$ is undecidable.
        (contradiction)
    \end{solution}

\end{enumerate}

\end{document}

